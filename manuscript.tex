% Options for packages loaded elsewhere
\PassOptionsToPackage{unicode}{hyperref}
\PassOptionsToPackage{hyphens}{url}
%
\documentclass[
]{article}
\usepackage{amsmath,amssymb}
\usepackage{lmodern}
\usepackage{iftex}
\ifPDFTeX
  \usepackage[T1]{fontenc}
  \usepackage[utf8]{inputenc}
  \usepackage{textcomp} % provide euro and other symbols
\else % if luatex or xetex
  \usepackage{unicode-math}
  \defaultfontfeatures{Scale=MatchLowercase}
  \defaultfontfeatures[\rmfamily]{Ligatures=TeX,Scale=1}
\fi
% Use upquote if available, for straight quotes in verbatim environments
\IfFileExists{upquote.sty}{\usepackage{upquote}}{}
\IfFileExists{microtype.sty}{% use microtype if available
  \usepackage[]{microtype}
  \UseMicrotypeSet[protrusion]{basicmath} % disable protrusion for tt fonts
}{}
\makeatletter
\@ifundefined{KOMAClassName}{% if non-KOMA class
  \IfFileExists{parskip.sty}{%
    \usepackage{parskip}
  }{% else
    \setlength{\parindent}{0pt}
    \setlength{\parskip}{6pt plus 2pt minus 1pt}}
}{% if KOMA class
  \KOMAoptions{parskip=half}}
\makeatother
\usepackage{xcolor}
\usepackage[margin=1in]{geometry}
\usepackage{longtable,booktabs,array}
\usepackage{calc} % for calculating minipage widths
% Correct order of tables after \paragraph or \subparagraph
\usepackage{etoolbox}
\makeatletter
\patchcmd\longtable{\par}{\if@noskipsec\mbox{}\fi\par}{}{}
\makeatother
% Allow footnotes in longtable head/foot
\IfFileExists{footnotehyper.sty}{\usepackage{footnotehyper}}{\usepackage{footnote}}
\makesavenoteenv{longtable}
\usepackage{graphicx}
\makeatletter
\def\maxwidth{\ifdim\Gin@nat@width>\linewidth\linewidth\else\Gin@nat@width\fi}
\def\maxheight{\ifdim\Gin@nat@height>\textheight\textheight\else\Gin@nat@height\fi}
\makeatother
% Scale images if necessary, so that they will not overflow the page
% margins by default, and it is still possible to overwrite the defaults
% using explicit options in \includegraphics[width, height, ...]{}
\setkeys{Gin}{width=\maxwidth,height=\maxheight,keepaspectratio}
% Set default figure placement to htbp
\makeatletter
\def\fps@figure{htbp}
\makeatother
\setlength{\emergencystretch}{3em} % prevent overfull lines
\providecommand{\tightlist}{%
  \setlength{\itemsep}{0pt}\setlength{\parskip}{0pt}}
\setcounter{secnumdepth}{5}
\newlength{\cslhangindent}
\setlength{\cslhangindent}{1.5em}
\newlength{\csllabelwidth}
\setlength{\csllabelwidth}{3em}
\newlength{\cslentryspacingunit} % times entry-spacing
\setlength{\cslentryspacingunit}{\parskip}
\newenvironment{CSLReferences}[2] % #1 hanging-ident, #2 entry spacing
 {% don't indent paragraphs
  \setlength{\parindent}{0pt}
  % turn on hanging indent if param 1 is 1
  \ifodd #1
  \let\oldpar\par
  \def\par{\hangindent=\cslhangindent\oldpar}
  \fi
  % set entry spacing
  \setlength{\parskip}{#2\cslentryspacingunit}
 }%
 {}
\usepackage{calc}
\newcommand{\CSLBlock}[1]{#1\hfill\break}
\newcommand{\CSLLeftMargin}[1]{\parbox[t]{\csllabelwidth}{#1}}
\newcommand{\CSLRightInline}[1]{\parbox[t]{\linewidth - \csllabelwidth}{#1}\break}
\newcommand{\CSLIndent}[1]{\hspace{\cslhangindent}#1}
\ifLuaTeX
  \usepackage{selnolig}  % disable illegal ligatures
\fi
\IfFileExists{bookmark.sty}{\usepackage{bookmark}}{\usepackage{hyperref}}
\IfFileExists{xurl.sty}{\usepackage{xurl}}{} % add URL line breaks if available
\urlstyle{same} % disable monospaced font for URLs
\hypersetup{
  pdftitle={Dike experiment},
  pdfauthor={Markus Bauer*, Jakob Huber, Johannes Kollmann},
  hidelinks,
  pdfcreator={LaTeX via pandoc}}

\title{Dike experiment}
\author{Markus Bauer*, Jakob Huber, Johannes Kollmann}
\date{}

\begin{document}
\maketitle

{
\setcounter{tocdepth}{2}
\tableofcontents
}
Restoration Ecology, TUM School of Life Sciences, Technical University of Munich, Germany

* Corresponding author: \href{mailto:markus1.bauer@tum.de}{\nolinkurl{markus1.bauer@tum.de}}

Open Research: Data and code are permanently available on Zenodo under: \url{https://doi.org/10.XXXX}.

\textbf{Max 8000 words}

\clearpage

\newpage

\hypertarget{abstract-max-300-words}{%
\section*{Abstract (max 300 words)}\label{abstract-max-300-words}}
\addcontentsline{toc}{section}{Abstract (max 300 words)}

\ref{fig:figure-1}

\hypertarget{keywords}{%
\section*{Keywords}\label{keywords}}
\addcontentsline{toc}{section}{Keywords}

dike grassland

embankment

species composition

environmental filter

\clearpage

\newpage

\hypertarget{introduction}{%
\section{Introduction}\label{introduction}}

``Ecological theory can point restoration toward important processes that need manipulation {[}\ldots{]}. However, for this information to be relevant, restoration ecology needs to employ evidence-based assessments {[}\ldots{]}'' (Suding 2011) --\textgreater{} No evidence for theory with our project

\begin{itemize}
\item
  Endangered grasslands
\item
  Predictability of restorations --\textgreater{} experiments, evidence-based restoration (cooke 2018)
\item
  Brudvig 2017 among restoration outcomes and in comparison to reference
\item
  Substrate modification to influence habitat filtering
\item
  Sowing a typical approach + priority effects + species pool
\item
  Introduce FCS, Recovery time, Persistence
\item
  Dike grasslands as case study
\item
  The aim of the study is
\end{itemize}

We asked following question to evaluate the restoration success after four years:

\begin{enumerate}
\def\labelenumi{\arabic{enumi}.}
\item
  How close is the vegetation to the reference state?
\item
  How strong differs the Favourable Conservation Status (FCS) among the seed mixture-substrate combinations?
\item
  How strong differs the persistence of the sown species among the seed mixture-substrate combinations?
\end{enumerate}

\clearpage

\newpage

\hypertarget{material-methods}{%
\section{Material \& Methods}\label{material-methods}}

\hypertarget{experimental-design}{%
\subsection*{Experimental design}\label{experimental-design}}
\addcontentsline{toc}{subsection}{Experimental design}

The seed mixture-substrate combination were tested with an experiment on a dike of the River Danube in SE Germany, which was established in March 2018 (Figure \ref{figure-1}), 314 m above sea level (asl); WGS84 (lat/lon), 48.83895, 12.88412). The climate is temperate-suboceanic with a mean annual temperature of 8.4 °C and precipitation of 984 mm (Deutscher Wetterdienst 2021). During the study, three exceptionally dry years (2018--2020) occurred (Climate Data Center of the German Meteorological Service 2022a; Climate Data Center of the German Meteorological Service 2022b), and three minor floods which did not reach the plots (\textbf{Bayerisches Landesamt für Umwelt {[}LfU{]}, 2021a, 2021b)} (Table \ref{table-s1}). Regional sand (0--4 mm) was used to lean the substrate and the soil was taken from the dike Is that right? and a with an excavator the substrates were mixed Is that right? and the plots prepared. The target vegetation were lowland hay meadows (Arrhenatherion elatioris, CM01A) and calcareous grassland (Cirsio-Brachypodion pinnati, DA01B code of the EuroVegChecklist, Mucina et al. (2016) ). The species pool of hay meadows and calcareous grasslands consisted of 55 and 58 species check numbers, respectively. The seeds were received from a commercial producer of autochtonous seeds. From the species pools, 20 species were randomly selected for each plot. The seed mixtures always contained seven grasses (60 wt\% of total seed mixture), three legumes (5\%) and ten herbs (35\%) (Table \ref{table-2})check. The community-weighted means (CWM) of functional traits differed between the seed mixture types (Table \ref{table-3}) more details and check. The management started with a cut without hay removal and a mowing height of 20 cm in August 2018, followed by `normal' deep cuts with hay removal in July 2019 and 2020. In October 2018, Bromus hordeaceus was seeded as a nursery grass to provide safe sites under drought conditions.

We used 288 plots of the size 2.0 m × 2.0 m which were distributed over the north and south exposition of the dike and arranged in six blocks (= replicates). The experiment used a split plot design combined with randomized complete block design (Figure \ref{figure-1}). The split plot was cuased by testing all 24 treatments on both sides of the dike. We tested three substrates with 0\%, 25\% and 50\% sand admixture and two soil depths (15 vs.~30 cm). Below the substrate, a drainage layer of 5 cm consisting of gravel (0--16 mm) was installed. The sand admixture changed the soil texture, increased the C/N ratio, but reduced the ratio of Calciumcarbonat, and hardly changed the pH (Table \ref{table-1}). The two seed mixture types were tested with two seeding densities (4 vs.~8 g m\^{}-2 ). The vegetation was surveyed in June or July 2018--2021 (\textbf{Braun-Blanquet, 1928/1964}). Establishment rates of species

\hypertarget{statistical-analyses}{%
\subsection*{Statistical analyses}\label{statistical-analyses}}
\addcontentsline{toc}{subsection}{Statistical analyses}

We performed all analyses in R (Version 4.2.2; R Core Team (2022)), with the functions `blme' (based on `lme4') for BLMM (Bates et al. 2015) \textbf{Chung, Rabe-Hesketh, Dorie, Gelman, \& Liu, 2013)}; and `DHARMa' for model evaluation (Hartig 2022).

We calculated Bayesian linear mixed-effects models (BLMM) with the random effect `plot' and used the restricted maximum-likelihood estimation (REML), the optimiser Nelder-Mead and, for the random effect, the Wishart prior. To identify the final model, we first reviewed the residual diagnostics of the candidate models and subsequently compared the remaining models using the Akaike information criterion adjusted for a small sample size (AICc) and chose the most parsimonious model (Appendix A4). Finally, we calculated the marginal and conditional coefficients of determination (R²m, R²c) and the 95\% confidence intervals of the response variables.

\clearpage

\newpage

\hypertarget{results}{%
\section{Results}\label{results}}

\clearpage

\newpage

\hypertarget{discussion}{%
\section{Discussion}\label{discussion}}

Text

\clearpage

\newpage

\hypertarget{conclusions}{%
\section{Conclusions}\label{conclusions}}

Text

\clearpage

\newpage

\hypertarget{acknowledgements}{%
\section*{Acknowledgements}\label{acknowledgements}}
\addcontentsline{toc}{section}{Acknowledgements}

We would like to thank our project partners Dr.~Markus Fischer, Frank Schuster, and Christoph Schwahn (WIGES GmbH) for numerous discussions on restoration and management of dike grasslands. Field work was supported by Clemens Berger and Uwe Kleber-Lerchbaumer (Wasserwirtschaftsamt Deggendorf). Mention Naturschutzbehörde and funding from WIGES. We thank Holger Paetsch, Simon Reith, Anna Ritter, Jakob Strak, Leonardo H. Teixeira, and Linda Weggler for assisting with the field surveys or soil analyses in 2018--2020. The German Federal Environmental Foundation (DBU) supported MB with a doctoral scholarship.

\hypertarget{author-contribution}{%
\section*{Author contribution}\label{author-contribution}}
\addcontentsline{toc}{section}{Author contribution}

JH and JK designed the experiment. JH did the surveys in the years 2018--2020 and MB in 2019 and 2021. MB did the analyses and wrote the manuscript. JK and JH critically revised the manuscript.

\hypertarget{open-research}{%
\section*{Open research}\label{open-research}}
\addcontentsline{toc}{section}{Open research}

Data and code is available on Zenodo: \url{https://doi.org/10.XXXX}

Model evaluation is stored on GitHub: \href{https://github.com/markus1bauer/2023_danube_dike_experiment/tree/main/markdown}{github.com/markus1bauer/2023\_danube\_dike\_experiment}

\hypertarget{funding}{%
\section*{Funding}\label{funding}}
\addcontentsline{toc}{section}{Funding}

MB is funded by a doctoral scholarship of the German Federal Environmental Foundation (DBU) (No.~20021/698). The establishment of the experiment and the vegetation surveys were financed by the WIGES GmbH in the years 2018--2020. funding number WIGES

\hypertarget{references}{%
\section*{References}\label{references}}
\addcontentsline{toc}{section}{References}

\hypertarget{refs}{}
\begin{CSLReferences}{1}{0}
\leavevmode\vadjust pre{\hypertarget{ref-lme4}{}}%
Bates, D., Machler, M., Bolker, B., \& Walker, S. 2015. \href{https://doi.org/10.18637/jss.v067.i01}{Fitting linear mixed-effects models using {\textbraceleft}lme4{\textbraceright}}. 67:

\leavevmode\vadjust pre{\hypertarget{ref-cdc2022temp}{}}%
Climate Data Center of the German Meteorological Service. 2022a. \href{https://cdc.dwd.de/portal/}{Annual station observations of air temperature at 2 m above ground in °c for germany. Version v21.3: Station metten}.

\leavevmode\vadjust pre{\hypertarget{ref-cdc2022prec}{}}%
Climate Data Center of the German Meteorological Service. 2022b. \href{https://cdc.dwd.de/portal/}{Annual station observations of precipitation in mm for germany. Version v21.3: Station metten}.

\leavevmode\vadjust pre{\hypertarget{ref-dwd2021}{}}%
Deutscher Wetterdienst. 2021. \href{https://www.dwd.de}{Langjähriges mittel der wetterstation metten 1981-2010}.

\leavevmode\vadjust pre{\hypertarget{ref-DHARMa}{}}%
Hartig, F. 2022. \href{https://CRAN.R-project.org/package=DHARMa}{DHARMa: Residual diagnostics for hierarchical (multi-level / mixed) regression models}.

\leavevmode\vadjust pre{\hypertarget{ref-mucina2016}{}}%
Mucina, L., Bültmann, H., Dierßen, K., Theurillat, J.-P., Raus, T., Čarni, A., Šumberová, K., Willner, W., Dengler, J., García, R.G., Chytrý, M., Hájek, M., Di Pietro, R., Iakushenko, D., Pallas, J., Daniëls, F.J.A., Bergmeier, E., Santos Guerra, A., Ermakov, N., Valachovič, M., Schaminée, J.H.J., Lysenko, T., Didukh, Y.P., Pignatti, S., Rodwell, J.S., Capelo, J., Weber, H.E., Solomeshch, A., Dimopoulos, P., Aguiar, C., Hennekens, S.M., \& Tichý, L. 2016. \href{https://doi.org/10.1111/avsc.12257}{Vegetation of Europe: hierarchical floristic classification system of vascular plant, bryophyte, lichen, and algal communities} (R. Peet, Ed.). \emph{Applied Vegetation Science} 19: 3--264.

\leavevmode\vadjust pre{\hypertarget{ref-base}{}}%
R Core Team. 2022. \href{https://www.R-project.org/}{R: A language and environment for statistical computing}.

\end{CSLReferences}

\clearpage

\newpage

\hypertarget{tables}{%
\section*{Tables}\label{tables}}
\addcontentsline{toc}{section}{Tables}

\hypertarget{table-1}{%
\subsection*{Table 1}\label{table-1}}
\addcontentsline{toc}{subsection}{Table 1}

\clearpage

\newpage

\hypertarget{table-2}{%
\subsection*{Table 2}\label{table-2}}
\addcontentsline{toc}{subsection}{Table 2}

\clearpage

\newpage

\hypertarget{figures}{%
\section*{Figures}\label{figures}}
\addcontentsline{toc}{section}{Figures}

See Figure \ref{fig:figure-1}

\includegraphics{outputs/figures/figure_1_map_tmap_300dpi_8x11cm.tiff}

\clearpage

\newpage

\hypertarget{figure-2}{%
\subsection*{Figure 2}\label{figure-2}}
\addcontentsline{toc}{subsection}{Figure 2}

\clearpage

\newpage

\hypertarget{figure-3}{%
\subsection*{Figure 3}\label{figure-3}}
\addcontentsline{toc}{subsection}{Figure 3}

\clearpage

\newpage

\hypertarget{figure-4}{%
\subsection*{Figure 4}\label{figure-4}}
\addcontentsline{toc}{subsection}{Figure 4}

\clearpage

\newpage

\hypertarget{figure-5}{%
\subsection*{Figure 5}\label{figure-5}}
\addcontentsline{toc}{subsection}{Figure 5}

Fig. 5:

\clearpage

\newpage

\hypertarget{figure-6}{%
\subsection*{Figure 6}\label{figure-6}}
\addcontentsline{toc}{subsection}{Figure 6}

Fig. 6:

\clearpage

\newpage

\hypertarget{supplementary-material}{%
\section*{Supplementary Material}\label{supplementary-material}}
\addcontentsline{toc}{section}{Supplementary Material}

\hypertarget{table-s1}{%
\subsection*{Table S1}\label{table-s1}}
\addcontentsline{toc}{subsection}{Table S1}

Three dry years (2018--2020) and three minor floods (2018, 2019, 2021) occurred during the study period. Annual temperature, precipitation and discharge of River Danube near the study sites (2017--2021), based on weather station Metten (mean, 1981--2010; ID, 3271; WGS84 (lat/lon), 48.85476 and 12.918911; CDC, 2022a, 2022b) and stream gauge Pfelling (ID, 10078000; WGS84 (lat/lon), 48.87975 and 12.74716; LfU, 2021a). HQ2 = Highest discharge with a probability of occurrence every second year. HSW = Highest water level for shipping.

\clearpage

\newpage

\hypertarget{table-s2}{%
\subsection*{Table S2}\label{table-s2}}
\addcontentsline{toc}{subsection}{Table S2}

\clearpage

\newpage

\hypertarget{table-s3}{%
\subsection*{Table S3}\label{table-s3}}
\addcontentsline{toc}{subsection}{Table S3}

\clearpage

\newpage

\hypertarget{session-info}{%
\subsection*{Session Info}\label{session-info}}
\addcontentsline{toc}{subsection}{Session Info}

\begin{verbatim}
## R version 4.2.2 (2022-10-31 ucrt)
## Platform: x86_64-w64-mingw32/x64 (64-bit)
## Running under: Windows 10 x64 (build 22621)
## 
## Matrix products: default
## 
## locale:
## [1] LC_COLLATE=German_Germany.utf8  LC_CTYPE=German_Germany.utf8   
## [3] LC_MONETARY=German_Germany.utf8 LC_NUMERIC=C                   
## [5] LC_TIME=German_Germany.utf8    
## 
## attached base packages:
## [1] stats     graphics  grDevices datasets  utils     methods   base     
## 
## other attached packages:
## [1] knitr_1.41    bookdown_0.31
## 
## loaded via a namespace (and not attached):
##  [1] here_1.0.1      rprojroot_2.0.3 digest_0.6.30   lifecycle_1.0.3
##  [5] magrittr_2.0.3  evaluate_0.19   highr_0.9       rlang_1.0.6    
##  [9] stringi_1.7.8   cli_3.4.1       renv_0.16.0     rstudioapi_0.14
## [13] vctrs_0.5.1     rmarkdown_2.18  tools_4.2.2     stringr_1.5.0  
## [17] glue_1.6.2      xfun_0.35       yaml_2.3.6      fastmap_1.1.0  
## [21] compiler_4.2.2  htmltools_0.5.3
\end{verbatim}

\end{document}
